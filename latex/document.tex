\documentclass{ltjsarticle}

% パッケージの読み込み
\usepackage{graphicx}  % 画像挿入用
\usepackage{amsmath}  % 数式用
\usepackage{hyperref} % ハイパーリンク用
\usepackage{geometry} % ページレイアウト調整用
\usepackage{fancyhdr}
\usepackage{caption}
\usepackage{multicol}
\usepackage{listings}
\usepackage{xcolor}
\usepackage[edges]{forest}
\usepackage{tcolorbox}
\usepackage{tikz}
\usepackage{listings,jvlisting}
\usetikzlibrary{arrows.meta, positioning}

% ページの余白を調整
\geometry{top=10mm,bottom=25mm,left=10mm,right=10mm}

% ハイパーリンクの色設定
\hypersetup{
    colorlinks=true,
    linkcolor=blue,
    filecolor=magenta,
    urlcolor=cyan
}

% ヘッダー・フッターの設定
\pagestyle{fancy}
\fancyhf{} % 既存のヘッダー・フッターをリセット
% \fancyfoot[R]{\includegraphics[width=2cm]{images/logo.png}} % フッター右下にロゴを配置
\fancyfoot[C]{\thepage}

\newtcbox{\code}[1][]{
  colback=gray!10!white,
  colframe=gray!20!white,
  boxrule=1pt,
  left=0mm,right=0mm,top=0mm,bottom=0mm,
  box align=base,
  nobeforeafter,
  fontupper=\ttfamily
}

\lstset{
  basicstyle={\ttfamily},
  identifierstyle={\small},
  commentstyle={\smallitshape},
  keywordstyle={\small\bfseries},
  ndkeywordstyle={\small},
  stringstyle={\small\ttfamily},
  frame={tb},
  breaklines=true,
  columns=[l]{fullflexible},
  numbers=left,
  xrightmargin=0pt,
xleftmargin=30pt,
numbersep=10pt,
  numberstyle={\scriptsize},
  stepnumber=1,
  lineskip=-0.5ex
}


% ドキュメント情報
\title{知能機械情報学 レポート課題1}
\author{03240258 機械情報工学科B4 有村 東真}
\date{2025年5月21日}

\newcommand{\imgwidthtwo}{0.36\linewidth}
\newcommand{\imgwidththree}{0.54\linewidth}


\begin{document}

% 表紙
\maketitle
\newpage

\setcounter{tocdepth}{3}
\tableofcontents % 目次

\newpage
\section{ソースコード}
本課題に取り組むにあたりpythonにおいて以下のコードを作成した
\begin{itemize}
  \item \code{hopfield.py}: Hopfield Networkの学習と想起の処理を記述したコード
  \item \code{image\_create.py}: 入力用の2値画像\code{.png}を作成するコード。配列として画像を定義する
  \item \code{image\_binarizer.py}: カラー画像を入力用の2値画像\code{.png}に変換するコード
  \item \code{image\_noise\_adder.py}: 2値画像に対して指定した割合でノイズを付加するコード
  \item \code{calc\_image\_distance.py}: 学習に使用した画像と想起された画像の差異を調べるコード
\end{itemize}
ソースコードは\href{https://github.com/ankorom0tim0ti/HopfieldNetwork}{Github}上で公開している。
\section{画像の学習と想起}
\subsection{1画像の学習と想起}
\subsubsection{入力画像}
\begin{figure}[h]
  \centering
  \begin{minipage}[b]{0.18\columnwidth}
      \centering
      \includegraphics[width=0.9\columnwidth]{../images/image_1-h5-w5.png}
      \caption{元画像}
  \end{minipage}
  \begin{minipage}[b]{0.18\columnwidth}
      \centering
      \includegraphics[width=0.9\columnwidth]{../images/image_1-h5-w5-noise0.05.png}
      \caption{ノイズ0.05}
  \end{minipage}
  \begin{minipage}[b]{0.18\columnwidth}
      \centering
      \includegraphics[width=0.9\columnwidth]{../images/image_1-h5-w5-noise0.1.png}
      \caption{ノイズ0.1}
  \end{minipage}
  \begin{minipage}[b]{0.18\columnwidth}
      \centering
      \includegraphics[width=0.9\columnwidth]{../images/image_1-h5-w5-noise0.15.png}
      \caption{ノイズ0.15}
  \end{minipage}
  \begin{minipage}[b]{0.18\columnwidth}
      \centering
      \includegraphics[width=0.9\columnwidth]{../images/image_1-h5-w5-noise0.2.png}
      \caption{ノイズ0.2}
  \end{minipage}
\end{figure}
\subsubsection{出力画像}
\begin{figure}[h]
  \centering
  \begin{minipage}[b]{0.18\columnwidth}
      \centering
      \includegraphics[width=0.9\columnwidth]{../images/image_1-h5-w5.png}
      \caption{元画像}
  \end{minipage}
  \begin{minipage}[b]{0.18\columnwidth}
      \centering
      \includegraphics[width=0.9\columnwidth]{../python/results/image_1-h5-w5-noise0.05-image_1-asyn-result-2025-05-27_10-28-01.png}
      \caption{ノイズ0.05}
  \end{minipage}
  \begin{minipage}[b]{0.18\columnwidth}
      \centering
      \includegraphics[width=0.9\columnwidth]{../python/results/image_1-h5-w5-noise0.1-image_1-asyn-result-2025-05-27_10-28-02.png}
      \caption{ノイズ0.1}
  \end{minipage}
  \begin{minipage}[b]{0.18\columnwidth}
      \centering
      \includegraphics[width=0.9\columnwidth]{../python/results/image_1-h5-w5-noise0.15-image_1-asyn-result-2025-05-27_10-28-02.png}
      \caption{ノイズ0.15}
  \end{minipage}
  \begin{minipage}[b]{0.18\columnwidth}
      \centering
      \includegraphics[width=0.9\columnwidth]{../python/results/image_1-h5-w5-noise0.2-image_1-asyn-result-2025-05-27_10-28-02.png}
      \caption{ノイズ0.2}
  \end{minipage}
\end{figure}
\subsubsection{結果}
基本的には何のノイズ付加率においても元画像を想起できている。この時ノイズ率0.05の場合において元画像から画像が変化していない。
\newpage
\subsection{想起性能調査}

\subsubsection{検証内容}
今回作成したHopfield型ネットワークの想起性能を検証するために、1, 2, 4, 6種類の画像を記憶させたネットワークに対して前項図1の画像に対してノイズを加えた画像を入力してその際の正答率や元画像に対する類似度を調査した。
それぞれの場合に記憶に使用した画像は以下の通りである。
\begin{figure}[h]
  \centering
  \begin{minipage}[b]{0.16\columnwidth}
      \centering
      \includegraphics[width=0.9\columnwidth]{../images/image_1-h5-w5.png}
      \caption{image\_1}
  \end{minipage}
  \begin{minipage}[b]{0.16\columnwidth}
      \centering
      \includegraphics[width=0.9\columnwidth]{../images/image_2-h5-w5.png}
      \caption{image\_2}
  \end{minipage}
  \begin{minipage}[b]{0.16\columnwidth}
      \centering
      \includegraphics[width=0.9\columnwidth]{../images/image_3-h5-w5.png}
      \caption{image\_3}
  \end{minipage}
  \begin{minipage}[b]{0.16\columnwidth}
      \centering
      \includegraphics[width=0.9\columnwidth]{../images/image_4-h5-w5.png}
      \caption{image\_4}
  \end{minipage}
  \begin{minipage}[b]{0.16\columnwidth}
      \centering
      \includegraphics[width=0.9\columnwidth]{../images/image_5-h5-w5.png}
      \caption{image\_5}
  \end{minipage}
  \begin{minipage}[b]{0.16\columnwidth}
      \centering
      \includegraphics[width=0.9\columnwidth]{../images/image_6-h5-w5.png}
      \caption{image\_6}
  \end{minipage}
\end{figure}

この際正答率は、想起した画像が元画像を完全に想起できた試行の全体の試行に対する割合を示し、類似度平均は全試行における想起した画像と元画像の間で異なっている画素の数の平均を示す。
\begin{center}
  \captionof{table}{正答率}
  \begin{tabular}{|l|r|r|r|r|r|r|r|r|r|r|r|r|} \hline
    学習画像/ノイズ率 & 0.05 & 0.1 & 0.15 & 0.2 & 0.3 & 0.4 & 0.5 & 0.6 & 0.7 & 0.8 & 0.9 & 1.0 \\ \hline
    1画像 & 100 & 100 & 100 & 100 & 100 & 95 & 80 & 95 & 65 & 90 & 55 & 75 \\
    2画像 & 100 & 95 & 100 & 90 & 100 & 90 & 100 & 100 & 85 & 100 & 55 & 70 \\
    4画像 & 100  & 100 & 85 & 100 & 90 & 100 & 85 & 80 & 85 & 95 & 25 & 35 \\
    6画像 & 100 & 95 & 100 & 85 & 80 & 80 & 95 & 85 & 60 & 100 & 10 & 0 \\ \hline
  \end{tabular}
  \captionof{table}{想起性能}
  \begin{tabular}{|l|r|r|r|r|r|r|r|r|r|r|r|r|} \hline
    学習画像/ノイズ率 & 0.05 & 0.1 & 0.15 & 0.2 & 0.3 & 0.4 & 0.5 & 0.6 & 0.7 & 0.8 & 0.9 & 1.0 \\ \hline
    1画像 & 0 & 0 & 0 & 0 & 0 & 0.05 & 0.2 & 0.05 & 0.4 & 0.1 & 6.35 & 3.85 \\
    2画像 & 0 & 0.05 & 0 & 0.1 & 0 & 0.1 & 0 & 0 & 0.15 & 0 & 6.45 & 6.3 \\
    4画像 & 0  & 0 & 0.15 & 0 & 0.1 & 0 & 0.15 & 0.2 & 0.15 & 0.1 & 13.75 & 11.45 \\
    6画像 & 0 & 0.05 & 0 & 0.15 & 0.2 & 0.2 & 0.05 & 0.15 & 3.4 & 0 & 13.7 & 11 \\ \hline
  \end{tabular}
\end{center}

\subsubsection{結果}
結果として、記憶画像数・ノイズ割合の増大に伴い想起性能が低下している傾向が見られる。しかし今回の実験で得られた結果はネットワークの性能を議論するには十分でないと考えられる。
定性的にはノイズ割合が1.0の時に画像の正答率は、必ず画像が想起できたとしても 1/記憶画像数 になるはずである。しかし今回の試行の結果を確認すると大局的には条件を満たしているように見られるが、依然として期待値通りの振る舞いをしていない。
これの原因として考えられるのは試行回数が十分でなかったということと画像のサイズが小さいことが考えられる。画像が小さいことでランダムにノイズが付加されて生成される画像が実際には別の画像に非常に似通った画像として出力されてそれが想起のプロセスに対して影響を与えたということが考えられる。
\newpage
\subsubsection{課題解答}
\begin{enumerate}
  \item 記憶画像を増やした際の想起性能の変化: 各記憶画像に対する想起性能の値の和は以下の通りである
  \begin{center}
  \begin{tabular}{|l|r|} \hline
    学習画像 & 想起性能の和 \\ \hline
    1画像 & 11.0 \\
    2画像 & 13.15 \\
    4画像 & 26.05 \\
    6画像 & 28.9 \\ \hline
  \end{tabular}
\end{center}
\end{enumerate}
\subsection{個別課題}
前項において入力画像が小さいことによるネットワークの想起性能に対する影響について言及した。今回はそれに関して実際に影響が出るかを検証した。
\end{document}